\documentclass{article}
\usepackage[utf8]{inputenc}
\usepackage{amsmath}
\usepackage{amssymb}
\usepackage{fullpage}
\usepackage{ragged2e}
\usepackage{url}
\title{SOEN6011 Problem1}
\author{Hetvi Shah (40089272)}
\date{05 July 2019}
\begin{document}
\noindent
\maketitle
\section{tan(x) function}
\subsection{Introduction}
\begin{itemize}
    \item The tangent function, denoted tan, is defined as the quotient of the sine function by the cosine function, and it is defined wherever the cosine function takes a nonzero value. In symbols:
\begin{center}
    $tan(x)=\frac{sin(x)}{cos(x)}$
\end{center}\hfill    
\end{itemize}
\subsection{Properties of tangent function}
\subsubsection{Domain and Range}
\begin{itemize}
    \item It can be said that all real numbers belong to the domain of the tangent function except the zeroes of the cosine function i.e ;
\end{itemize}
% \begin{equation}
%     \begin{align}
        
%     \end{align}
% \end{equation}
\begin{center}
$D_f=\frac{R}{(2k+1)(\frac{\pi}{2})}$ ,$ k \in R$
\end{center}\hfill
% $D_f=\frac{2R}{(2k+1)$\pi$}$
\begin{itemize}
    \item Tangent function takes all the values from -$\infty$ to + $\infty$ as its argument x passes through an interval of the length $\pi$, therefore the range,
\end{itemize}
\begin{center}
    $f(D)=R$ 
\end{center}\hfill
\subsubsection{Zeros of the tangent function}
\begin{itemize}
    \item The zeroes of the tangent are determined by the zeroes of the sine function in the numerator.
\end{itemize}
\subsubsection{ Parity and periodicity}
\begin{itemize}
    \item Tangent function is a odd function.
\end{itemize}
\begin{center}
    $ f(-x) = tan(-x)=-tan(x)=-f(x)$
\end{center}\hfill
   \begin{itemize}
    \item The tangent is periodic function with the period  p = $\pi$
\end{itemize}
\subsection{Reference}
\begin{enumerate}
    \item  \url{http://www.nabla.hr/TF-TrigFunctionsD3.htm}
\end{enumerate}
\end{document}
